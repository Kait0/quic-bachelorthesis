% !TeX root = ../main.tex
\chapter{\abstractname}

TCP has been the most prevalent connection establishment protocol in the Internet for a long time.
Since TCP does not support encryption inherently it later got extended with SSL/TLS.
This introduced an additional overhead in connection establishment time but ensured backwards compatibility.
This overhead was of less concern in the past since bandwidth was the limiting factor of the Internet.
With increasing bandwidth of the Internet the focus shifted back to reducing round trips as connection speed is becoming a limiting factor which can not be improved because the speed of light is constant.
Over the last years attempts where made to improve TCP such as TCP Fast Open \cite{Link:TCPFastOpen}.
They did not see widespread adoption.
In 2012 Google started developing their own Connection Protocol called Quick UDP Internet Connections (QUIC).
Due to having control over both client and server infrastructures (Chrome, Android, Youtube.com ,Google.com, ...) they were able to deploy their protocol at a large scale and in 2017 as reported by Google an approximately 7\% of the Internet traffic was using the QUIC protocol \cite{DBLP:conf/sigcomm/LangleyRWVKZYKS17}.
The QUIC protocol is currently going through a standardisation process by the IETF \cite{Link:ietfQuic}.
This thesis aims to evaluate the connection latency of the Google QUIC protocol as it is deployed on production servers.
To achieve this we wrote two measurement test to evaluate the connection establishment times of QUIC and TCP/TLS.
Using these programs we collected data to compare both protocols from two vantage points in Munich Germany.
We also measured how many websites already adopted the new TLS 1.3 and how the protocol compares to TLS 1.2 and QUIC.
Furthermore we measured the impact IPv6 had on the connection establishment times of these protocols.
We find that QUIC connects faster than TLS 1.2 and TLS 1.3 in nearly all cases but the degree depended on the measurement environment.
TLS 1.3 connected faster than TLS 1.2 in almost all cases and IPv6 had little to no impact on the connection times.