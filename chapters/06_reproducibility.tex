% !TeX root = ../main.tex
\chapter{Reproducibility Considerations}\label{chapter:Reproducibility}

\section{Installation of the programs}
The first part will be a tutorial on how to set up the different libraries and programs we used to do our tests.
All measurements were done using a cable connection to avoid delays coming from wireless connections.
For the analysis we used \textit{Python 3.6} with Anaconda Jupyter Notebook.
The notebook files are available at \cite{Link:pythonFiles}. 
Some things such as the hardcoded paths might have to be adjusted to draw the individual plots.
Table \ref{tab:figures} shows which figure was drawn using which notebook.

\subsection{LS-QUIC client}
(Part of this information comes from the LSQUIC readme)
This tutorial is made for Ubuntu 16.04 LTS (code is compiled with gcc version 5.4.0)
The first step is to download the program from the git repository:
\begin{lstlisting}
cd $HOME
git clone https://github.com/Kait0/lsquic-client.git
\end{lstlisting}

Change to the directory:
\begin{lstlisting}
cd lsquic-client
git checkout master (if you are not already on it)
\end{lstlisting}

To build LSQUIC you need CMake, zlib, libevent and BoringSSL (Google QUIC uses the BoringSSL crypto):
\begin{lstlisting}
sudo apt-get update
sudo apt-get install cmake
sudo apt-get install zlib1g-dev
sudo apt-get install libevent-dev
\end{lstlisting}

Installing BoringSSL is a bit more complicated as you have to build it yourself.
To build BoringSSL you will need to install the Go language.
\begin{lstlisting}
sudo apt-get install golang
cd .. (if you are still in the lsquic-client folder)
git clone https://boringssl.googlesource.com/boringssl
cd boringssl
git checkout chromium-stable
cmake -DCMAKE_BUILD_TYPE=Release . && make 
(we want the maximum performance so we turn on optimisations)
BORINGSSL_SOURCE=$PWD
mkdir -p $HOME/tmp/boringssl-libs
cd $HOME/tmp/boringssl-libs
ln -s $BORINGSSL_SOURCE/ssl/libssl.a
ln -s $BORINGSSL_SOURCE/crypto/libcrypto.a
\end{lstlisting}

We will also need a crypto that supports TLS 1.3.
In our measurements we used OpenSSL 1.1.1.
You can download OpenSSL from their webpage:
\begin{lstlisting}
cd $HOME
wget https://www.openssl.org/source/openssl-1.1.1-pre8.tar.gz
tar -xvzf openssl-1.1.1-pre8.tar.gz
cd openssl-1.1.1-pre8
./config --release
make
make test (optional)
sudo make install
\end{lstlisting}

Now that we have the library we can link it to lsquic and build the lsquic library (again we want maximum performance so we compile in release mode)
\begin{lstlisting}
cd $HOME/lsquic-client
(The next two lines are one line.)
cmake -DBORINGSSL_INCLUDE=$BORINGSSL_SOURCE/include 
-DBORINGSSL_LIB=$HOME/tmp/boringssl-libs -DDEVEL_MODE=0 .
make
\end{lstlisting}

You can test if the installation worked (program should print the server response)
Be aware that QUIC builds upon UDP so your firewall needs to have that enabled in order to get a working connection:
\begin{lstlisting}
./quic_perf -s www.google.com:443 -p /
\end{lstlisting}

\subsection{tls\_perf}

To install \texttt{tls\_perf} we need to install a version of curl that supports TLS 1.3 (and Openssl 1.1.1 which we installed earlier)
\begin{lstlisting}
cd $HOME
wget https://curl.haxx.se/download/curl-7.61.0.tar.gz
tar -xvzf curl-7.61.0.tar.gz
cd curl-7.61.0
./configure --disable-threaded-resolver 
(we want the non-threaded DNS resolver since it's slightly faster)
make
make test (optional)
sudo make install
sudo ldconfig (update libraries so that we get the new OpenSSL)
\end{lstlisting}

Now that we have all the libraries we can install \texttt{tls\_perf}:
\begin{lstlisting}
cd $HOME
git clone https://github.com/Kait0/tls_perf.git
cd tls_perf
make
\end{lstlisting}

You can test the installation with e.g. :
\begin{lstlisting}
./tls_perf -u www.youtube.com -p 443
\end{lstlisting}

Both programs we used for measurements are now installed.

To collect data with them we used a simple bash script:

\begin{lstlisting}
#!/bin/bash
while IFS='' read -r line || [[ -n "$line" ]]; do
	cd $HOME/lsquic-client
    ./quic_perf -t -s $line:443 -p / >> $HOME/output.csv
	cd $HOME/tls_perf
	./tls_perf -u $line -p 443 >> $HOME/output.csv
    ./tls_perf -3 -u $line -p 443 >> $HOME/output.csv
done < "$1"
\end{lstlisting}

It takes a text file as an argument that should contain the list of websites you want to test.
The websites need to be separated by a line break \verb|\n|.
Make sure there are no \verb|\r| at the end of each line because the script won't work if that's the case.
The script than calls every website with the 3 different protocols and redirects the stdout output into the file output.csv .
The header we put in the output.csv file is:
\begin{lstlisting}
DnsLookupTime;TimeOfMeasurement;Url;Path;Ip;Port;
ConnectionEstablishmentTime;HttpResponse;Protocol;TcpHandshakeTimeOptional
\end{lstlisting}

The lists of websites we tested against are available at \cite{Link:WebsiteLists}.\\

To execute the script periodically we used the program cron:
\begin{lstlisting}
crontab -e
(add this line to execute the script every hour)
0 * * * * $HOME/measureScript websites-29-06.txt
\end{lstlisting}

\section{Plots}

All CSV files needed to draw the plots are stored in \cite{Link:CSVData} and described in \ref{section:dataset}

\begin{table}[!htb]
\centering
\caption{Mapping of Notebooks and Figures}
\begin{tabular}{| c | c |}
\hline
Path to Notebook & Figure\\
\hline	
\multirow{16}{*}{\textasciitilde /data/plot\_QUIC.ipynb} & 4.1\\
& 4.2 \\
& 4.3 \\
& 4.4 \\
& 4.13 \\
& 4.14 \\
& 4.15 \\
& 4.16 \\
& 4.17 \\
& 4.18 \\
& 4.19 \\
& 4.20 \\
& 4.21 \\
& 4.22 \\
& 4.23 \\
& 4.24 \\
\hline
\multirow{7}{*}{\textasciitilde /data/plot\_TLS13.ipynb} & 4.5 \\
& 4.6 \\
& 4.7 \\
& 4.8 \\
& 4.9 \\
& 4.10 \\
& 4.11 \\
& 4.12 \\
\hline	
\end{tabular}
\label{tab:figures}
\end{table}