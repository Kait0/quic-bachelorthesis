% !TeX root = ../main.tex
% Add the above to each chapter to make compiling the PDF easier in some editors.
\chapter{Introduction}\label{chapter:introduction}

\section{Motivation}
Google QUIC is an Internet protocol developed by Google with the aim to replace the commonly used TCP + TLS protocol.
As the name suggests one of it's goals is to enable faster connection establishments.
Google presented this protocol at SIGCOMM 2017 \cite{DBLP:conf/sigcomm/LangleyRWVKZYKS17} stating that they deployed QUIC on their servers and clients at an Internet scale and that an estimated of 7\% of the Internet traffic is using QUIC.
One goal of this thesis is to analyse the connection latency of QUIC as it is deployed in the Internet and compare it to the TCP + TLS protocol.
The Internet Engineering Task Force also started working on standardizing the QUIC protocol \cite{Link:ietfQuic}.
In this thesis we did not study IETF QUIC and focused on Google QUIC as it is used in production scenarios.\\
Another interesting development in the security protocol area is TLS version 1.3 which is currently an Internet-Draft \cite{ietf-tls-tls13-28}.
One goal of this thesis is to determine the penetration of TLS 1.3 in the Internet and to evaluate it's latency compared to it's predecessor TLS 1.2 and Google QUIC.\\
IPv6 has become an Internet Standard in 2017 \cite{Link:IPv6Internet} after being a Draft Standard since 1998 \cite{Link:IPv6Draft} and it has seen increasing adoption over the last years according to Google statistics \cite{Link:IPv6Adoption}.
Another goal of this thesis is to find out if IPv6 has any influence on the connection establishment times of Google QUIC or TCP/TLS.

\newpage

\section{Research Questions}

Several questions were researched in this work to measure and compare the connection latency of Google QUIC
as well as the new TLS 1.3 protocol.
The thesis engaged in following research Questions:

\begin{QandA}

   \item [RQ 1:] How many websites adopted TLS 1.3 already?
         \begin{answered}
         TLS 1.3 is currently an Internet-Draft \cite{ietf-tls-tls13-28}. Multiple libraries and websites already started to implement/deploy it.
         To compare QUIC to TLS 1.3 we need a list of websites that already support it.
         \end{answered}
         
   \item [RQ 2:] How much faster/slower does TCP/TLS 1.3 connect compared to TCP/TLS 1.2?
   		\begin{answered}
         TLS 1.3 aims to improve TLS 1.2 in terms of security and connection speed.
         We want to determine how much faster TLS 1.3 connects.
         \end{answered}
         
   \item [RQ 3:] How much faster/slower does QUIC connect compared to TCP/TLS 1.2?
         \begin{answered}
         QUIC aims to replace the TCP/TLS protocol.
         We are interested to see how much latency reduction QUIC gains compared to TLS 1.2. 
         \end{answered}
        
   \item [RQ 4:] How much faster/slower does QUIC connect compared to TCP/TLS 1.3?
   		\begin{answered}
        TCP/TLS 1.3 will likely become the new standard for encryption replacing TCP/TLS 1.2.
        If QUIC wants to replace TCP it needs to outperform TCP/TLS 1.3 as well.
        \end{answered}
\end{QandA}

\newpage

\section{Research Contributions}

The analysis and research in this thesis contribute to the performance evaluation of the Internet protocols Google QUIC, TCP/TLS 1.2 and TCP/TLS 1.3. Furthermore we analysed the adoption of TLS 1.3 and the influence of IPv6 on the protocols.

Our contributions are:
\begin{enumerate}
\item We developed one program called \texttt{quic\_perf} to measure the connection establishment times of Google QUIC.
The DNS lookup code of this program was merged in the LSQUIC test client.
We also created another program called \texttt{tls\_perf} to measure TCP/TLS 1.2 and TCP/TLS 1.3.
They are available at  \cite{Link:tls_perfGithub} and \cite{Link:quic_perfGithub}
\item We measured the adoption of TLS 1.3
In Alexa Top 1000 at least 10,7\% of websites adopted the protocol.
In Alexa Top 100.000 at least 15,9\% of websites adopted the protocol.
For more details see \ref{section:target_selection}.
\item We compared the latency of TLS 1.3 vs TLS 1.2. 
TLS 1.3 gains a latency reduction of \textasciitilde 10 ms over TLS 1.2  \textasciitilde 80\% of the time in our measurements. IPv6 had similar results compared to IPv4 , see \ref{section:TLS1_1.2_vs_Tls_1.3}.
\item We compared the latency of QUIC vs TLS 1.2.
QUIC had a latency reduction of \textasciitilde 30 ms over TLS 1.2 \textasciitilde 60\% of the time and an advantage of \textasciitilde 10 ms \textasciitilde 40\% of the time in one of the test environments.
On the second test environment QUIC had an advantage of \textasciitilde 5 ms \textasciitilde 40\% of the time and \textasciitilde 20 ms \textasciitilde 60\% of the time.
Also on the second test environment QUIC gains a \textasciitilde 20 ms advantage \textasciitilde 90 \% of the time with IPv6.
For more details see chapter \ref{section:measurement_setup} and \ref{QUIC_vs_TLS_1.2}.
\item We compared the latency of QUIC vs TLS 1.3.
QUIC connects \textasciitilde 10 ms faster than TLS 1.3 \textasciitilde 90\% of the time in one test environment and \textasciitilde 2 ms  \textasciitilde 99\% of the time on the second test environment.
IPv6 had similar results compared to IPv4.
For more details see \ref{section:QUIC_vs_TLS_1.3}.
\end{enumerate}

As with other work this thesis also has limitations which are discussed in Chapter \ref{chapter:conlusion} together with the conclusion.
Several lessons we learned while writing the thesis are explained in chapter \ref{chapter:lessons_learned} and further research opportunities are also outlined in chapter \ref{chapter:conlusion}.\\
Chapter \ref{chapter:Reproducibility} describes how to reproduce the results shown in this work.\\
Our methodology is described in chapter \ref{chapter:methodology} an evaluation of our result is done in chapter \ref{chapter_performance_evaluation}.